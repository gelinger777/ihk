\section{Qualitätsanforderungen}

    Die Qualitätsanforderungen an das Produkt wurden durch den Auftraggeber vorgegeben. Diese beinhalten folgende Punkte:

    \begin{itemize}
        \item Erweiterbarkeit
        \item Übliche Softwarewartung
        \item Benutzerfreundlichkeit
        \item Performance
        \item Stablilität / Zuverlässigkeit
        \item Vollständigkeit
    \end{itemize}
%
    Auf die Erweiterbarkeit wird hohen Wert gegelgt. Dies wirkt sich unter anderem auf die anderen Qualitätsmerkmale aus. Außerdem soll es dadurch in Zukunft möglich sein eine Bearbeitung aus einer anderer Hand zu ermöglichen.
    \\ \\
    Die Übliche Softwarewartung zeichenet sich mit kommentierten und strukturierten Qullcode ab. Die Produktfunktionen sollen im Code erkennbar sein.
    \\ \\
    Die Benutzerfreundlichkeit ist ebenfalls wichtig. Der Kunde legt Wert auf eine
    selbsterklärende und intuitive Oberfläche.
    \\ \\
    Durch Performance soll eine angemessene Datenverarbeitungsgeschwindigkeit sichergestellt werden. Diese sollte sich im angemessenen Rahmen befinden, indem keine lange Wartezeiten bei der Verwendung des Produkts auffallen.
    \\ \\
    Unerlässlich ist die Stabilität bzw. Zuverlässigkeit, welche eine stetige Erreichbarkeit unserer Server vorrausgesetzt und beim Produkt selber eine niedrige bis keine Fehlertoleranz erzielt.
    \\ \\
    Die Vollständigkeit wird hoch geschätzt, da alle Synergien übereinstimmen müssen, damit die Software erfolgreich verwendet werden kann.