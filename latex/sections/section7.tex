\section{Qualitätsanforderungen}
    Die Qualitätsanforderungen an das Produkt wurden durch den Auftraggeber vorgegeben. Diese beinhalten folgende Punkte:

    %
    \begin{center}
        \begin{tabular}{ |c|c|c|c|c| }
             \hline
                                & sehr gut & gut & normal & nicht relevant \\
             \hline
                \multicolumn{1}{|l|}{\textbf{Vollständigkeit}}  & x & & & \\
             \hline
                \multicolumn{1}{|l|}{\textbf{Zuverlässigkeit}}  & x & & & \\
             \hline
                \multicolumn{1}{|l|}{\textbf{Benutzbarkeit}}    & x & & & \\
             \hline
                \multicolumn{1}{|l|}{\textbf{Effizienz}}        & & & x & \\
             \hline
                \multicolumn{1}{|l|}{\textbf{Übertragbarkeit}}  & & & & x \\
             \hline
                \multicolumn{1}{|l|}{\textbf{Performance}}      & & x & & \\
             \hline
                \multicolumn{1}{|l|}{\textbf{Erweiterbarkeit}}  & x & & & \\
             \hline
                \multicolumn{1}{|l|}{\textbf{Wartbarkeit}}      & x & & & \\
             \hline
        \end{tabular}
    \end{center}
    %
    Die Vollständigkeit wird hoch geschätzt, welche durch das erfüllen aller Musskriterien verwirklicht wird, wodurch die Software erfolgreich verwendet werden kann.//
    \\ \\
    Unerlässlich ist die Stabilität bzw. Zuverlässigkeit, welche eine stetige Erreichbarkeit unserer Server vorausgesetzt und beim Produkt selber eine hohe Fehlertoleranz erzielt.
    \\ \\
    Die Benutzbarkeit ist ebenfalls wichtig. Der Auftragsteller legt Wert auf eine selbsterklärende und intuitive Oberfläche.
    \\ \\
    Die Effizienz wird normal Gewichtet. Prozessorleistung, Arbeitsspeicher oder hohes Datenvolumen wird hier nicht beansprucht.
    \\ \\
    Die Übertragtragbarkeit wird anhand unterschiedliche Anzeigeflächen definiert, d.h. von Mobil- bis Desktopansicht. Dies ist nicht Bestandteil des Pflichtenhefts. Für die anspruchsvolle Gestaltung auf unterschiedelichen Anzeigeflächen wird die Creativity Abteilung verantwortlich sein.
    \\ \\
    Durch Performance soll eine angemessene Datenverarbeitungsgeschwindigkeit sichergestellt werden. Diese sollte sich im angemessenen Rahmen befinden, indem keine lange Wartezeiten bei der Verwendung des Produkts anfallen.
    \\ \\
    Auf die Erweiterbarkeit wird hoher Wert gelegt. Dies wirkt sich unter anderem auf die anderen Qualitätsmerkmale aus. Außerdem soll es in Zukunft möglich sein, dass das Produkt durch andere Entwickler weiterentwickelt werden kann.
    \\ \\
    Die Wartbarkeit zeigt sich im kommentierten und strukturierten Quellcode ab. Die Produktfunktionen sollen im Code erkennbar sein.


